\chapter{线性空间与线性变换}
线性空间是集合空间与$n$维向量空间的推广。线性变换反映了现行空间中元素的一种联系。线性空间和线性变换的概念比较抽象。

几何方法与代数方法的融合是数学自身的需要和数学统一性的体现,也是处理工程问题的有力手段。学习本章时一定要注意思想的来源,并联系所讨论的问题在平面和空间指教坐标系中的原型,将抽象的代数概念几何直观化。

\section{线性空间}
线性空间是矩阵论最基本的概念之一,是对各种具体线性系统的一种统一的抽象。下面首先介绍基础概念。
\subsection{集合、数域与映射}
设给定$n$个集合$A_{1},A_{2},\dots,A_{n}$,由$A_{1},A_{2},\dots,A_{n}$的所有元素组成的集合称为这些集合的\textbf{并集},记为$A_{1}\cup{A_{2}}\dots\cup{A_{n}}$。
由$A_{1},A_{2},\dots,A_{n}$的公共元素组成的集合称为这些集合的\textbf{交集},记为$A_{1}\cap{A_{2}}\dots\cap{A_{n}}$。

设$A,B$是两个集合,由所有属于$A$但不属于$B$的元素组成的集合称为集合$A$与$B$的\textbf{差},记作$A-B$。

设$A,B$是两个集合,集合$A\times{B}=\left\{\left(a,b\right)\left|a\in{A},b\in{B}\right.\right\}$称为$A$与$B$\textbf{积}。

\begin{definition}
    设$P$至少包含一个非零数的数集,如果$P$中任意两个数的和、差、积、商(分母不为零)仍属于$P$,称数集$P$为一个数域。
\end{definition}

显然,全体整数集不构成数域。全体有理数集$\mathbf{Q}$,全体实数集$\mathbf{R}$,全体复数集$\mathbf{C}$都构成数域,其中实数域$\mathbf{R}$和复数域$\mathbf{C}$是工程上较常用的两个数域。

\begin{definition}
    设$A,B$是两个非空集合,$A$到$B$的一个\textbf{映射} $T$是指一个对应法则,通过该法则,集合$A$中的任一元素$x$,都有集合$B$中唯一确定的元素$y$与之相对应,记作$T:x\rightarrow{y}$或者$T(x)=y$,$y$则称为$x$在映像$T$下的\textbf{像}, $x$称为$y$在映射$T$下的原像。
    集合$A$的所有元素的像的集合记作$T(A)=\left\{T(x)\left|x\in{A}\right.\right\}$。
\end{definition}

\begin{definition}
    设$T$是集合$A$到$B$的一个映射,如果对任意的$x_{1},x_{2}\in{A}$,当$x_{1}\neq{x_{2}}$时,有$T(x_{1})\neq{T(x_{2})}$,称$T$是\textbf{单射}。
    如果对任意的$y\in{B}$,有$x\in{A}$,使得$T(x)=y$,称$T$是\textbf{满射}。
    如果$T$既是单射又是满射,称为\textbf{一一对应},又称为\textbf{双射}。
\end{definition}

\begin{example}
    实数域$\mathbf{R}$上的$n\times{n}$阶矩阵全体$\mathbf{R}^{n\times{n}}$,定义
    \begin{eqnarray}
        T_{1}\left(\mathbf{A}\right)=\det\left(\mathbf{A}\right),T_{2}\left(\mathbf{A}\right)=a\mathbf{I}_{n},T_{3}\left(\mathbf{A}\right)=\mathbf{A}+\mathbf{I}_{n}\nonumber
    \end{eqnarray}
\end{example}
其中$\mathbf{A}\in\mathbf{R}^{n\times{n}}$,$a\in\mathbf{R}$是常数,$\det\mathbf{A}$表示矩阵$\mathbf{A}$的行列式,$\mathbf{I}_{n}$是$n$阶单位矩阵,则$T_{1}$是$\mathbf{R}^{n\times{n}}$到$\mathbf{R}$的满射,但不是单射;
$T_{2}$是$\mathbf{R}^{n\times{n}}$到$\mathbf{R}^{n\times{n}}$的单射,但不是满射;
$T_{3}$是$\mathbf{R}^{n\times{n}}$到$\mathbf{R}^{n\times{n}}$的双射。
\subsection{线性空间的定义与性质}
\begin{definition}
    设$P$是一个数域,$V$是一个非空集合,定义集合$V\times{V}$到$V$上的加法‘+’及集合$P\times{V}$到$V$上数乘‘.’两种映射,且这两种映射是封闭的,
    即运算后的结果仍然在$V$中,如果这两种线性运算对任意的$\alpha,\beta.\gamma\in{V}$和$k,l\in{P}$,满足下面8条运算率,那么称集合$V$为数域$P$上的\textbf{线性空间}:
    \begin{enumerate}[label=(\arabic*)]
        \item 加法交换律:$\alpha+\beta=\beta+\alpha$;
        \item 加法结合律:$(\alpha+\beta)+\gamma=\alpha+(\beta+\gamma)$;
        \item 零元素存在:即存在任意元素$\alpha$,存在$0$,使得$\alpha+0=\alpha$;
        \item 负元素存在:即存在任意元素$\alpha$,存在$-\alpha$,使得$\alpha+(-\alpha)=0$;
        \item 数乘分配率:$k\cdot\left(\alpha+\beta\right)=k\cdot\alpha+k\cdot\beta$;
        \item 分配率:$(k+l)\cdot\alpha=k\cdot\alpha+l\cdot\beta$;
        \item 数乘结合律:$\left(kl\right)\cdot\alpha=k\cdot\left(l\cdot\alpha\right)$;
        \item 单位元存在:存在元素$1$,使得$1\cdot\alpha=\alpha$。
    \end{enumerate}

    如果$V$是$P$上的线性空间,称$V$中的元素为向量,$P$中的元素为纯两。当$P$为实数域$\mathbf{R}$(复数域$\mathbf{C}$)时,称$V$为\textbf{实(复)线性空间}。
\end{definition}

\begin{remark}
    数乘符号‘.’通常省略不写。
\end{remark}

\begin{example}
    数域$P$上的全体$n$维向量构成的集合$P^{n}$按通常的加法与数乘,构成线性空间$P^{n}$.
    特别地,实数域$\mathbf{R}$上的$n$维向量全体,按照向量加法与向量的数乘运算构成线性空间$\mathbf{R}^{n}$,
    复数域$\mathbf{C}$上的$n$维向量全体,按向量加法与向量的数乘构成线性空间$\mathbf{C}^{n}$。
\end{example}

\begin{example}
    实数域$\mathbf{R}$上的$m\times{n}$阶矩阵全体,按矩阵的加法和数乘,构成实数域$\mathbf{R}$上的线性空间$\mathbf{R}^{m\times{n}}$。
\end{example}

\begin{example}
    \label{example:1-14}
    设$\mathbf{R}^{+}$表示全体正实数集合,对任意的$x,y\in\mathbf{R}^{+}$,定义加法$\oplus$与数乘$\circ$分别为:
    \begin{eqnarray}
        x\oplus{y}=ab,k\circ{x}=a^{k}\left(a,b\in{\mathbf{R}^{+}},k\in\mathbf{R}\right)\nonumber
    \end{eqnarray}
\end{example}
可验证$\mathbf{R}^{+}$对加法$\oplus$和数乘$\circ$构成实数域上的线性空间。

\begin{example}
    数域$P$上多项式全体按照多项式的加法,以及数与多项式的乘法构成$P$上的线性空间,记作$P\left[x\right]$。
\end{example}

\begin{example}
    数域$P$上次数小于等于$n$的一元多项式再加上零多项式按照多项式的加法,以及数与多项式的乘法构成$P$上的线性空间,记作$P_{n}\left[x\right]$。
\end{example}

\begin{example}
    区间$\left[a,b\right]$上全体连续实值函数全体按通常函数的加法和数与函数的乘法构成线性空间,记作$C\left[a,b\right]$。
\end{example}

\begin{example}
    其次线性方程组$\mathbf{Ax}=\mathbf{0}$的所有解的集合构成实数域$\mathbf{R}$上的线性空间,称为矩阵$\mathbf{A}$的\textbf{零空间}(或\textbf{核空间}),记作$\text{Ker}\left(\mathbf{A}\right)$。即
    \begin{eqnarray}
        \text{Ker}\left(\mathbf{A}\right)&=\left\{\mathbf{x}\in{\mathbf{R}^{n}}\left|\mathbf{Ax}=\mathbf{0},\mathbf{A}\in\mathbf{R}^{m\times{n}}\right.\right\}
    \end{eqnarray}
\end{example}

非齐次线性方程组$\mathbf{Ax=b}$的所有解的集合一般不构成实数域$\mathbf{R}$上的线性空间。因为该集合对加法运算不封闭。

\begin{example}
    给定矩阵$\mathbf{A}\in{\mathbf{R}^{m\times{n}}}$,集合$\left\{\mathbf{y}\left|\mathbf{y=Ax,x\in{R}}^{n}\right.\right\}$构成实数域$\mathbf{R}$上的线性空间,称为矩阵$\mathbf{A}$的\textbf{值域},也称为$\mathbf{A}$的\textbf{像}(空间),记作$\mathbf{R(A)}$。
\end{example}

\begin{example}
    集合$V_{1}=\left\{\mathbf{x}\left|\mathbf{x}=\left(x_{1},x_{2},0\right)^{T},x_{1},x_{2}\in\mathbf{R}\right.\right\}$是一个线性空间。但集合$V_{2}=$\\
    $\left\{\mathbf{x}\left|\mathbf{x}=\left(x_{1},x_{2},1\right)^{T},x_{1},x_{2}\in\mathbf{R}\right.\right\}$不是一个线性空间,因为$V_{2}$对加法运算不封闭。
\end{example}

\begin{remark}
    \begin{enumerate}[label=(\arabic*)]
        \item 线性空间不能离开某一数域来定义。实际上,对于不同数域,同一个集合构成的线性空间会不同,甚至一种能成为线性空间而另一种不能成为线性空间。例如$\mathbf{C}$作为$\mathbf{C}$上的线性空间与$\mathbf{C}$作为$\mathbf{R}$上的线性空间是不同的,$\mathbf{R}^{n}$在复数域$\mathbf{C}$上的不构成线性空间。
        \item 数域$P$中的运算是具体的四则运算,而$V$中所定义的加法运算和数乘运算可以是熟悉一般的运算,也可以是各种特殊的运算,如例\autoref{example:1-14}。
        \item 唯一性一般较显然,封闭性通常需要证明。
        \item 线性空间中的元素可以使向量、矩阵、多项式、函数等。
    \end{enumerate}
\end{remark}

\subsection{线性空间的基、维数与坐标}
\begin{definition}
    设$V$是数域$P$上的线性空间。$\alpha_{1},\alpha_{2},\cdots,\alpha_{n}\in{V}$,$\lambda_{1},\lambda_{2},\cdots,\lambda_{n}\in{P}$,则称
    \begin{eqnarray}
        \lambda_{1}\alpha_{1}+\lambda_{2}\alpha_{2}+\cdots+\lambda_{n}\alpha_{n}\nonumber
    \end{eqnarray}
    为向量组$\alpha_{1},\alpha_{2},\cdots,\alpha_{n}$的一个线性组合。如果不存在一组不全为零的常数$\lambda_{1},\lambda_{2},\cdots,\lambda_{n}\in{P}$,使得
    \begin{eqnarray}
        \lambda_{1}\alpha_{1}+\lambda_{2}\alpha_{2}+\cdots+\lambda_{n}\alpha_{n}=\mathbf{0}\nonumber        
    \end{eqnarray}
    则称向量组$\alpha_{1},\alpha_{2},\cdots,\alpha_{n}$\textbf{线性相关},否则称向量组$\alpha_{1},\alpha_{2},\cdots,\alpha_{n}$\textbf{线性无关}。
\end{definition}

\begin{example}
    讨论$\mathbf{R}^{2\times{2}}$中,向量组
    \begin{eqnarray}
        \alpha_{1}=\left[\begin{array}{cc}
            -1 & 5\\
            1& 12
        \end{array}\right],
        \alpha_{2}=\left[\begin{array}{cc}
            5 & 5\\
            -2& 4
        \end{array}\right],
        \alpha_{3}=\left[\begin{array}{cc}
            4 & -2\\
            5& -12
        \end{array}\right],
        \alpha_{4}=\left[\begin{array}{cc}
            5 & -3\\
            2& -8
        \end{array}\right]\nonumber
    \end{eqnarray}
    的线性相关性。
\end{example}

\begin{solution}
    设$\lambda_{1},\lambda_{2},\lambda_{3},\lambda_{4}\in{\mathbf{R}}$,使得
    \begin{eqnarray}
        \lambda_{1}\alpha_{1}+\lambda_{2}\alpha_{2}+\lambda_{3}\alpha_{3}+\lambda_{4}\alpha_{4}=\mathbf{0}\nonumber
    \end{eqnarray}
    即
    \begin{eqnarray}
        \lambda_{1}\left[\begin{array}{cc}
            -1 & 5\\
            1& 12
        \end{array}\right]+\lambda_{2}\left[\begin{array}{cc}
            5 & 5\\
            -2& 4
        \end{array}\right]+\lambda_{3}\left[\begin{array}{cc}
            4 & -2\\
            5& -12
        \end{array}\right]+\lambda_{4}\left[\begin{array}{cc}
            5 & -3\\
            2& -8
        \end{array}\right]=\left[\begin{array}{cc}
            0 & 0\\
            0 & 0
        \end{array}\right]\nonumber
    \end{eqnarray}
    由矩阵相等的定义,可以得到以下的线性方程组:
    \begin{eqnarray}
        \begin{cases}
            -\lambda_{1}+5\lambda_{2}+4\lambda_{3}+5\lambda_{4}=0\\
            5\lambda_{1}+5\lambda_{2}-2\lambda_{3}-3\lambda_{4}=0\\
            \lambda_{1}-2\lambda_{2}+5\lambda_{3}+2\lambda_{4}=0\\
            12\lambda_{1}+4\lambda_{2}-12\lambda_{3}-8\lambda_{4}=0
        \end{cases}\label{equation:12}
    \end{eqnarray}

    由于方程组(\ref{equation:12})的系数行列式为零,所以有非零解,故向量$\alpha_{1},\alpha_{2},\alpha_{3},\alpha_{4}$线性相关。
\end{solution}

\begin{definition}
    设线性空间$V$是数域$P$上的线性空间,$V$满足以下条件的向量组$\alpha_{1},\alpha_{2},\cdots,\alpha_{n}$称为线性空间$V$的一组\textbf{基}。
    \begin{enumerate}[label=(\arabic*)]
        \item $\alpha_{1},\alpha_{2},\cdots,\alpha_{n}$线性无关;
        \item 线性空间$V$中任意向量都能由$\alpha_{1},\alpha_{2},\cdots,\alpha_{n}$线性表示。
    \end{enumerate}
    基中的向量个数$n$称为线性空间$V$的维数,记作$\text{dim}(V)=n$。
\end{definition}

\begin{remark}

    \begin{enumerate}[label=(\arabic*)]
        \item 基是线性空间$V$的最大线性无关组;$V$的维数是基中所包含元素的个数;
        \item 基不是唯一的,但不同的基所包含元素个数相等;
        \item 线性空间不一定是有限维的,如$P[x],C[a,b]$是无限维的,这时基的元素也是无限的。
    \end{enumerate}
\end{remark}

\begin{example}
    对线性空间$P^{n}$,令$e_{i}$为第$i$个分量为$1$,其他分量为零的向量,则$e_{1},e_{2}.\cdots,e_{n}$是线性空间$P^{n}$的一组基(自然基)。
\end{example}

\begin{example}
    对线性空间$\mathbf{R}^{m\times{n}}$,令$\mathbf{E}_{i,j}$为这样一个$m\times{n}$阶矩阵,其$(i,j)$元素为$1$,其余元素为零。显然这样的矩阵共有$mn$个,构成了线性空间$\mathbf{R}^{m\times{n}}$的一组基。
\end{example}

\begin{example}
    数域$P$上次数小于等于$n$的一元多项式线性空间
    \begin{eqnarray}
        P_{n}[x]=\left\{p(x)\left|p(x)=a_{0}+a_{1}x+a_{2}x^{2}+\cdots+a_{n}x^{n}\right.\right\}\nonumber
    \end{eqnarray}
    则$1,x,\cdots,x^{n}$是一组基(自然基),$P_{n}[x]$是$n+1$维的。
\end{example}

\begin{example}
    对齐次线性方程组$\mathbf{A}_{m\times{n}}\mathbf{x=0}$的零空间$\text{Ker}(\mathbf{A})$,任意一组基础解系即为一组基。
\end{example}

\begin{theorem}
    设向量组$\alpha_{1},\alpha_{2},\cdots,\alpha_{n}$是线性空间$V^{n}$中的一组基,则$V^{n}$任意向量都可以唯一表示称向量组$\alpha_{1},\alpha_{2},\cdots,\alpha_{n}$的线性组合,即$V^{n}$可以表示为
    \begin{eqnarray}
        V^{n}=\left\{\alpha\left|\alpha=x_{1}\alpha_{1}+x_{2}\alpha_{2}+\cdots+x_{n}\alpha_{n},x_{1},x_{2},\cdots,x_{n}\in{P}\right.\right\}\nonumber
    \end{eqnarray}
\end{theorem}

\begin{definition}
    设$\alpha_{1},\alpha_{2},\cdots,\alpha_{n}$是线性空间$V^{n}$的一组基,对$V^{n}$的任意向量$\alpha$,有
    \begin{eqnarray}
        \alpha=x_{1}\alpha_{1}+x_{2}\alpha_{2}+\cdots+x_{n}\alpha_{n}\nonumber
    \end{eqnarray}
    则称$\left(x_{1},x_{2},\cdots,x_{n}\right)^{T}$为向量$\alpha$在基$\alpha_{1},\alpha_{2},\cdots,\alpha_{n}$下的\textbf{坐标}。
\end{definition}

建立坐标后,线性空间$V^{n}$中的向量$\alpha$与向量$\left(x_{1},x_{2},\cdots,x_{n}\right)$建立了一一对应的关系。
设$\alpha_{1},\alpha_{2},\cdots,\alpha_{n}$是线性空间$V^{n}$的一组基,对任意的$\alpha,\beta\in{V^{n}}$,有
\begin{eqnarray}
    \alpha=x_{1}\alpha_{1}+x_{2}\alpha_{2}+\cdots+x_{n}\alpha_{n},\beta=y_{1}\alpha_{1}+y_{2}\alpha_{2}+\cdots+y_{n}\alpha_{n}\nonumber
\end{eqnarray}

即$\alpha,\beta\in{V^{n}}$在基$\alpha_{1},\alpha_{2},\cdots,\alpha_{n}$下的坐标分别为$\left(x_{1},x_{2},\cdots,x_{n}\right)^{T}$和$\left(y_{1},y_{2},\cdots,y_{n}\right)^{T}$,则有
\begin{eqnarray}
    \begin{aligned}
        \alpha+\beta&=(x_{1}+y_{1})\alpha_{1}+(x_{2}+y_{2})\alpha_{2}+\cdots+(x_{n}+y_{n})\alpha_{n}\\
        k\alpha&=kx_{1}\alpha_{1}+kx_{2}\alpha_{2}+\cdots+kx_{n}\alpha_{n}
    \end{aligned}
\end{eqnarray}

即$\alpha+\beta$在基$\alpha_{1},\alpha_{2},\cdots,\alpha_{n}$下的坐标为$\left(x_{1}+y_{1},x_{2}+y_{2},\cdots,x_{n}+y_{n}\right)^{T}$,$k\alpha$在基$\alpha_{1},\alpha_{2},\cdots,\alpha_{n}$下的坐标为$\left(kx_{1},kx_{2},\cdots,kx_{n}\right)^{T}$。

这种一一对应保持了线性运算的对应关系,可以说线性空间$V^{n}$与$P^{n}$具有相同的结构,称为\textbf{同构}。因为这种同构关系,只需要主要研究元素为向量的线性空间$P^{n}$。

下面研究基改变时坐标变化规律。

设$\alpha_{1},\alpha_{2},\cdots,\alpha_{n}$和$\beta_{1},\beta_{2},\cdots,\beta_{n}$是线性空间$V^{n}$的两组基。由于两者都是基,所以两者可以相互线性表示,即
\begin{eqnarray}
    \begin{cases}
        \beta_{1}&=a_{11}\alpha_{1}+a_{21}\alpha_{2}+\cdots+a_{n1}\alpha_{n}\\
        \beta_{2}&=a_{12}\alpha_{1}+a_{22}\alpha_{2}+\cdots+a_{n2}\alpha_{n}\\
        &\vdots\\
        \beta_{1}&=a_{1n}\alpha_{1}+a_{2n}\alpha_{2}+\cdots+a_{nn}\alpha_{n}
    \end{cases}\label{equation:1-4}
\end{eqnarray}

设矩阵

\begin{eqnarray}
    \textbf{A}&=\left[\begin{array}{cccc}
        a_{11}&a_{12}&\cdots&a_{n1}\\
        a_{21}&a_{22}&\cdots&a_{n2}\\
        \vdots&\vdots&\ddots&\vdots\\
        a_{n1}&a_{n2}&\cdots&a_{nn}
    \end{array}\right]
\end{eqnarray}

则式(\ref{equation:1-4})可以简记为
\begin{eqnarray}
    \left(\beta_{1},\beta_{2},\cdots,\beta_{n}\right)=\left(\alpha_{1},\alpha_{2},\cdots,\alpha_{n}\right)\mathbf{A}\label{equation:1-6}
\end{eqnarray}

\begin{theorem}
    设$\mathbf{A}$是线性空间$V^{n}$由基$\alpha_{1},\alpha_{2},\cdots,\alpha_{n}$到基$\beta_{1},\beta_{2},\cdots,\beta_{n}$的过渡矩阵,则$\mathbf{A}$是可逆的,且由基$\beta_{1},\beta_{2},\cdots,\beta_{n}$到$\alpha_{1},\alpha_{2},\cdots,\alpha_{n}$的过渡矩阵为$\mathbf{A}^{-1}$。
\end{theorem}

\begin{proof}
    由已知条件,有
    \begin{eqnarray}
        \left(\beta_{1},\beta_{2},\cdots,\beta_{n}\right)=\left(\alpha_{1},\alpha_{2},\cdots,\alpha_{n}\right)\mathbf{A}\nonumber
    \end{eqnarray}
    于是
    \begin{eqnarray}
        \left|\left(\beta_{1},\beta_{2},\cdots,\beta_{n}\right)\right|=\left|\left(\alpha_{1},\alpha_{2},\cdots,\alpha_{n}\right)\right|\cdot\left|\mathbf{A}\right|\nonumber
    \end{eqnarray}
    因为向量组$\alpha_{1},\alpha_{2},\cdots,\alpha_{n}$与$\beta_{1},\beta_{2},\cdots,\beta_{n}$均为基,所以有
    \begin{eqnarray}
        \left|\left(\beta_{1},\beta_{2},\cdots,\beta_{n}\right)\right|\neq{0},\left|\left(\alpha_{1},\alpha_{2},\cdots,\alpha_{n}\right)\right|\neq{0}\nonumber
    \end{eqnarray}
    所以$\left|\mathbf{A}\right|\neq{0}$,即矩阵$\left|\mathbf{A}\right|$是可逆的。(\ref{equation:1-6})式两边右乘$\mathbf{A}^{-1}$,有
    \begin{eqnarray}
        \left(\beta_{1},\beta_{2},\cdots,\beta_{n}\right)\mathbf{A}^{-1}=\left(\alpha_{1},\alpha_{2},\cdots,\alpha_{n}\right)\nonumber
    \end{eqnarray}
    即基$\beta_{1},\beta_{2},\cdots,\beta_{n}$到基$\alpha_{1},\alpha_{2},\cdots,\alpha_{n}$的过渡矩阵为$\mathbf{A}^{-1}$。
\end{proof}

\begin{theorem}
    设$V^{n}$是数域$P$上的一个线性空间,$\mathbf{A}$是由基$\alpha_{1},\alpha_{2},\cdots,\alpha_{n}$到基$\beta_{1},\beta_{2},\dots,\beta_{n}$的过渡矩阵,
    向量$\alpha$关于$\alpha_{1},\alpha_{2},\cdots,\alpha_{n}$的坐标为$\mathbf{x}=\left(x_{1},x_{2},\cdots,x_{n}\right)^{T}$,关于基$\beta_{1},\beta_{2},\cdots,\beta_{n}$的坐标为
    $\mathbf{y}=\left(y_{1},y_{2},\cdots,y_{n}\right)^{T}$,则$\mathbf{x=Ay}$,或者$\mathbf{y=A^{-1}x}$。
\end{theorem}

\begin{proof}
    因为
    \begin{eqnarray}
        \begin{aligned}
            \alpha&=y_{1}\beta_{1}+y_{2}\beta_{2}+\cdots+y_{n}\beta_{n}\\
            &=\left(\beta_{1},\beta_{2},\cdots,\beta_{n}\right)\mathbf{y}\\
            &=\left(\alpha_{1},\alpha_{2},\cdots,\alpha_{n}\right)\mathbf{y}
        \end{aligned}\nonumber
    \end{eqnarray}
    
    又因为
    \begin{eqnarray}
        \alpha=\left(\alpha_{1},\alpha_{2},\cdots,\alpha_{n}\right)\mathbf{x}\nonumber
    \end{eqnarray}

    由$\alpha_{1},\alpha_{2},\cdots,\alpha_{n}$是一组基,得
    \begin{eqnarray}
        \mathbf{x=Ay}\nonumber
    \end{eqnarray}
    证毕。
\end{proof}

\begin{example}
    已知$\alpha_{1},\alpha_{2},\alpha_{3}$线性空间$V^{3}$的一组基,向量组$\beta_{1},\beta_{2},\beta_{3}$满足
    \begin{eqnarray}
        \beta_{1}+\beta_{3}=\alpha_{1}+\alpha_{2}+\alpha_{3},\beta_{1}+\beta_{3}=\alpha_{2}+\alpha_{3},\beta_{2}+\beta_{3}=\alpha_{1}+\alpha_{3}\nonumber
    \end{eqnarray}

    \begin{enumerate}[label=(\arabic*)]
        \item 证明向量组$\beta_{1},\beta_{2},\beta_{3}$也是一组基;
        \item 求由基$\beta_{1},\beta_{2},\beta_{2}$到基$\alpha_{1},\alpha_{2},\alpha_{3}$的过渡矩阵;
        \item 求向量$\alpha=\alpha_{1}+2\alpha_{2}-\alpha_{3}$在基$\beta_{1},\beta_{2},\beta_{3}$下的坐标。
    \end{enumerate}
\end{example}

\begin{solution}
    \begin{enumerate}[label=(\arabic*)]
        \item 由已知条件,有
        \begin{eqnarray}
            \left(\beta_{1},\beta_{2},\beta_{3}\right)\left[\begin{array}{ccc}
                1&1&0\\
                0&1&1\\
                1&0&1
            \end{array}\right]=\left(\alpha_{1},\alpha_{2},\alpha_{3}\right)\left[\begin{array}{ccc}
                1&0&1\\
                1&1&0\\
                1&1&1
            \end{array}\right]\label{exmaple:1-16}
        \end{eqnarray}
        由于上述表达式两边矩阵均可逆,所以$\beta_{1},\beta_{2},\beta_{3}$线性无关,即$\beta_{1},\beta_{2},\beta_{3}$也是一组基。
        \item 由表达式\ref{exmaple:1-16},有
        \begin{eqnarray}
            \left(\alpha_{1},\alpha_{2},\alpha_{3}\right)=\left(\beta_{1},\beta_{2},\beta_{3}\right)\left[\begin{array}{ccc}
                1&0&1\\
                1&1&0\\
                1&1&1
            \end{array}\right]\left[\begin{array}{ccc}
                1&1&0\\
                0&1&1\\
                1&0&1
            \end{array}\right]^{-1}=\left(\beta_{1},\beta_{2},\beta_{3}\right)\left[
                \begin{array}{ccc}
                    0&1&0\\
                    -1&-1&2\\
                    1&0&0
                \end{array}
            \right]\nonumber
        \end{eqnarray}
        所以由基$\beta_{1},\beta_{2},\beta_{3}$到基$\alpha_{1},\alpha_{2},\alpha_{3}$的过渡矩阵为$\left[
            \begin{array}{ccc}
                0&1&0\\
                -1&-1&2\\
                1&0&0
            \end{array}
        \right]$。
        \item 由题意,有
        \begin{eqnarray}
            \alpha&=\alpha_{1}+2\alpha_{2}-\alpha_{3}=\left(\alpha_{1},\alpha_{2},\alpha_{3}\right)\left[\begin{array}{c}
                1\\
                2\\
                -1                
            \end{array}\right]\nonumber\\
            &=\left(\beta_{1},\beta_{2},\beta_{3}\right)\left[\begin{array}{ccc}
                0&1&0\\
                -1&-1&2\\
                1&0&0
            \end{array}\right]\left[\begin{array}{ccc}
                1\\
                2\\
                -1
            \end{array}\right]
            &=\left(\beta_{1},\beta_{2},\beta_{3}\right)\left[\begin{array}{ccc}
                2\\
                -5\\
                1
            \end{array}\right]\nonumber
        \end{eqnarray}
        所以向量$\alpha=\alpha_{1}+2\alpha_{2}-\alpha_{3}$在基$\beta_{1},\beta_{2},\beta_{3}$下的坐标为$\left(2,-5,1\right)$。
    \end{enumerate}
\end{solution}
\subsection{子空间定义}
整体有时太庞大,所以通过“部分来获知整体”。对线性空间的研究亦是如此,通过对线性空间的局部进行深入的研究,能够更加深刻地揭示整个线性空间的结构。
\begin{definition}
    设$V$是数域$P$上的线性空间,$S$为$V$的一个非空子集,若$S$对$V$已有的加法和乘法运算也构成一个线性空间,称$S$为$V$的一个\textbf{线性子空间},简称\textbf{子空间}。
\end{definition}

\begin{theorem}
    线性空间$V$的一个非空子集$S$构成子空间的充分必要条件为$S$对$V$已有的线性运算封闭。
\end{theorem}

\begin{proof}
    必要性显然成立。下面证明充分性。

    已知$S$对加法与数乘运算封闭,则线性空间定义的运算律(1),(2),(5)$\sim$(8)均成立。
    又$\mathbf{0}\in{S},-\alpha\in{S}$,故运算律(3)(4)成立,即$S$构成线性空间。

    证毕。
\end{proof}

\section{线性变换}
\section{应用案例}
